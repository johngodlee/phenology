% latex table generated in R 4.0.2 by xtable 1.8-4 package
% Fri Oct 30 14:10:04 2020
\begin{table}[H]
\centering
\begin{tabular}{ccccccc}
  \hline
Cluster & N sites & Richness & MAP & Diurnal $\delta$T & Species & Indicator value \\ 
  \hline
 &  &  &  &  & \textit{Pterocarpus angolensis} & 0.294 \\ 
  1 & 416 & 15(7) & 1040(199.8) & 14(1.6) & \textit{Diplorhynchus condylocarpon} & 0.265 \\ 
   &  &  &  &  & \textit{Brachystegia spiciformis} & 0.252 \\ 
   \hline
 &  &  &  &  & \textit{Brachystegia boehmii} & 0.795 \\ 
  2 & 135 & 16(5) & 1051(165.1) & 13(1.5) & \textit{Psuedolachnostylis maprouneifolia} & 0.240 \\ 
   &  &  &  &  & \textit{Uapaca kirkiana} & 0.224 \\ 
   \hline
 &  &  &  &  & \textit{Julbernardia paniculata} & 0.717 \\ 
  3 & 153 & 15(7) & 989(153.1) & 14(1.4) & \textit{Psuedolachnostylis maprouneifolia} & 0.272 \\ 
   &  &  &  &  & \textit{Diplorhynchus condylocarpon} & 0.228 \\ 
  \end{tabular}
\caption{Climatic information and Dufrene-Legendre indicator species analysis for the vegetation type clusters identified by the PAM algorithm. The three species per cluster with the highest indicator values are shown along with other key statistics for each cluster. MAP (Mean Annual Precipitation) and Diurnal $\delta$T are reported as the mean and 1 standard deviation in parentheses. Species richness is reported as the median and the interquartile range in parentheses.} 
\label{clust_summ}
\end{table}

