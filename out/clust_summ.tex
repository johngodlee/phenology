% latex table generated in R 4.0.2 by xtable 1.8-4 package
% Tue Nov 10 17:13:14 2020
\begin{table}[H]
\centering
\begin{tabular}{ccccccc}
  \hline
Cluster & N sites & Richness & MAP & Diurnal $\delta$T & Species & Indicator value \\ 
  \hline
 &  &  &  &  & \textit{Julbernardia paniculata} & 0.712 \\ 
  1 & 91 & 13(6) & 966(139.7) & 14(1.3) & \textit{Psuedolachnostylis maprouneifolia} & 0.222 \\ 
   &  &  &  &  & \textit{Pericopsis angolensis} & 0.209 \\ 
   \hline
 &  &  &  &  & \textit{Brachystegia boehmii} & 0.764 \\ 
  2 & 127 & 16(6) & 1054(162.5) & 13(1.5) & \textit{Psuedolachnostylis maprouneifolia} & 0.234 \\ 
   &  &  &  &  & \textit{Uapaca kirkiana} & 0.227 \\ 
   \hline
 &  &  &  &  & \textit{Pterocarpus angolensis} & 0.333 \\ 
  3 & 487 & 15(7) & 1037(195.9) & 14(1.6) & \textit{Brachystegia spiciformis} & 0.318 \\ 
   &  &  &  &  & \textit{Diplorhynchus condylocarpon} & 0.298 \\ 
  \end{tabular}
\caption{Climatic information and Dufrene-Legendre indicator species analysis for the vegetation type clusters identified by the PAM algorithm, based on basal area weighted species abundances. The three species per cluster with the highest indicator values are shown along with other key statistics for each cluster. MAP (Mean Annual Precipitation) and Diurnal $\delta$T are reported as the mean and 1 standard deviation in parentheses. Species richness is reported as the median and the interquartile range in parentheses.} 
\label{clust_summ}
\end{table}

